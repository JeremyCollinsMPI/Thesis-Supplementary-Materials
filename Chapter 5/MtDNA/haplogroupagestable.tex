\documentclass{article}

\usepackage{longtable}
\usepackage[margin=0.5in]{geometry}

\begin{document}


%\begin{table}[h] 
%\centering
\begin{longtable}{| p{.20\textwidth} | p{.15\textwidth} | p{.03\textwidth} | p{.05\textwidth} | p{.05\textwidth} | p{.10\textwidth} | p{.05\textwidth} | p{.03\textwidth} | p{.038\textwidth} | p{.035\textwidth} |p{.05\textwidth} } 
  \hline
Population & Language & ISO & Lat & Long & Family & N & H & Onset & Coda \\ 
  \hline
Libyan Tuareg & Tamahaq & thv & 22.8 & 5.2 & Afro-Asiatic & 129 & 0.61 & 1 & 1 \\ 
  Berbers in Morocco & Median of Morocco Berber & shi & 31 & -5 & Afro-Asiatic & 125 & 0.37 & 3 & 3 \\ 
  Algeria & Median of Algeria Berber & kab & 36.5 & 5 & Afro-Asiatic & 82 & 0.26 & 2 & 2 \\ 
  Tunisia & Tunisian Arabic & aeb & 36 & 10 & Afro-Asiatic & 83 & 0.27 & 3 & 2 \\ 
  Berbers in Egypt & Egyptian Arabic & arz & 30 & 30 & Afro-Asiatic & 71 & 0.01 & 1 & 2 \\ 
  Syrians & Syrian Arabic & apc & 33 & 37 & Afro-Asiatic & 28 & 0.18 & 3 & 2 \\ 
  Iraq & Arabic & arb & 33 & 44 & Afro-Asiatic & 206 & 0.20 & 1 & 1 \\ 
   \hline
\caption{Your caption here} % needs to go inside longtable environment
\label{tab:myfirstlongtable}
\end{longtable}
%\end{table} 

Table \ref{tab:myfirstlongtable} shows my first longtable.
\end{document}
